% Options for packages loaded elsewhere
\PassOptionsToPackage{unicode}{hyperref}
\PassOptionsToPackage{hyphens}{url}
%
\documentclass[
]{article}
\usepackage{amsmath,amssymb}
\usepackage{lmodern}
\usepackage{ifxetex,ifluatex}
\ifnum 0\ifxetex 1\fi\ifluatex 1\fi=0 % if pdftex
  \usepackage[T1]{fontenc}
  \usepackage[utf8]{inputenc}
  \usepackage{textcomp} % provide euro and other symbols
\else % if luatex or xetex
  \usepackage{unicode-math}
  \defaultfontfeatures{Scale=MatchLowercase}
  \defaultfontfeatures[\rmfamily]{Ligatures=TeX,Scale=1}
\fi
% Use upquote if available, for straight quotes in verbatim environments
\IfFileExists{upquote.sty}{\usepackage{upquote}}{}
\IfFileExists{microtype.sty}{% use microtype if available
  \usepackage[]{microtype}
  \UseMicrotypeSet[protrusion]{basicmath} % disable protrusion for tt fonts
}{}
\makeatletter
\@ifundefined{KOMAClassName}{% if non-KOMA class
  \IfFileExists{parskip.sty}{%
    \usepackage{parskip}
  }{% else
    \setlength{\parindent}{0pt}
    \setlength{\parskip}{6pt plus 2pt minus 1pt}}
}{% if KOMA class
  \KOMAoptions{parskip=half}}
\makeatother
\usepackage{xcolor}
\IfFileExists{xurl.sty}{\usepackage{xurl}}{} % add URL line breaks if available
\IfFileExists{bookmark.sty}{\usepackage{bookmark}}{\usepackage{hyperref}}
\hypersetup{
  pdftitle={TP2 Econometría Avanzada},
  pdfauthor={Casiano, Denys; Daboin, Carlos y Quispe, Anzony},
  hidelinks,
  pdfcreator={LaTeX via pandoc}}
\urlstyle{same} % disable monospaced font for URLs
\usepackage[left=2.5cm,right=2.5cm,top=1.5cm,bottom=2cm]{geometry}
\usepackage{graphicx}
\makeatletter
\def\maxwidth{\ifdim\Gin@nat@width>\linewidth\linewidth\else\Gin@nat@width\fi}
\def\maxheight{\ifdim\Gin@nat@height>\textheight\textheight\else\Gin@nat@height\fi}
\makeatother
% Scale images if necessary, so that they will not overflow the page
% margins by default, and it is still possible to overwrite the defaults
% using explicit options in \includegraphics[width, height, ...]{}
\setkeys{Gin}{width=\maxwidth,height=\maxheight,keepaspectratio}
% Set default figure placement to htbp
\makeatletter
\def\fps@figure{htbp}
\makeatother
\setlength{\emergencystretch}{3em} % prevent overfull lines
\providecommand{\tightlist}{%
  \setlength{\itemsep}{0pt}\setlength{\parskip}{0pt}}
\setcounter{secnumdepth}{-\maxdimen} % remove section numbering
\usepackage{booktabs} \usepackage{longtable} \usepackage{array} \usepackage{multirow} \usepackage{wrapfig} \usepackage{float} \usepackage{dcolumn} \floatplacement{figure}{H}
<<<<<<< HEAD
=======
\ifluatex
  \usepackage{selnolig}  % disable illegal ligatures
\fi
>>>>>>> 757fa87904969bc08d68db32967ed273e724e447

\title{TP2 Econometría Avanzada}
\author{Casiano, Denys; Daboin, Carlos y Quispe, Anzony}
\date{2022-04-12}

\begin{document}
\maketitle

<<<<<<< HEAD
\hypertarget{notaciuxf3n}{%
\subsection{Notación}\label{notaciuxf3n}}
=======
\hypertarget{notacion}{%
\subsection{Notacion}\label{notacion}}
>>>>>>> 757fa87904969bc08d68db32967ed273e724e447

Las variables consideradas en los modelos disctuidos a continuación son
las siguientes:

\begin{itemize}
\tightlist
\item
  \(R_{it}\) : la tasa de criminalidad del condado i en el periodo t.
\item
  \(X'_{it}\) : la matriz de controles del rendimiento de la oportunidad
  legal.
\item
  \(P'_{it}\) : la matriz de variables del sistema judicial.
\item
  \(\alpha_{i}\) son los efectos fijos.
\item
  \(\epsilon_{it}\) : el termino de error.
\end{itemize}

\hypertarget{discusion-del-estimador-between-be}{%
\subsection{1. Discusion del estimador between
(BE)}\label{discusion-del-estimador-between-be}}

<<<<<<< HEAD
El estimador between (BE) básicamente un OLS de corte transversal fijado
con el promedio histórico de las variables a nivel sujeto (condado), por
lo que no controla ningún tipo de heterogeneidad no-observada entre
estos. Si las características no observables estuvieran correlacionadas
con \((X'_{it},P'_{it})\) el estimador BE produciría estimaciones
sesgadas. La estimación BE presentada en el paper sólo sería válida si
\((X'_{it},P'_{it})\) fueran ortogonales tanto a \(\alpha_i\) como a
\(\epsilon_{it}\).

Los coeficientes de las variables de justicia criminal \(P_A, P_C\) y
\(S\) cumplen con el signo negativo sugerido por el modelo económico de
crimen en nuestra réplica del modelo BE (ver Tabla 1), mientras que el
coeficiente de \(P_P\) no. Sólo los parámetros \(P_A\) y \(P_C\) son
signficativos. Según los resultados de este modelo, la probabilidad de
detención y la probabilidad de condena son significativas para explicar
reducción de la tasa de criminalidad.

Finalmente, las estimaciones de los parámetros en valor absoluto cumplen
con el ordenamiento de los efectos disuasorios del modelo de crimen
donde \(|P_A|>|P_C|>|P_P|\).
=======
El modelo between al no considerar la heterogeneidad no observada del
condado produce estimadores similares a las estimaciones de corte
transversal. Esto genera un problema, ya que si las características no
observables están correlacionadas con \((X'_{it},P'_{it})\) se generarán
estimaciones sesgadas y por tanto inconsistentes. Dado el modelo con
transformación \emph{between} presentado en el paper, el estimador solo
será consistente si \((X'_{it},P'_{it})\) son ortogonales tanto a
\(\alpha_i\) como a \(\epsilon_{it}\).

Los resultados del modelo muestran que las estimaciones para las
variables de justicia criminal (\(P_A, P_C, S\)) cumplen con el signo
negativo (sugerido por el modelo económico de crimen), a excepción de la
variable \(P_P\). Solo los parámetros \(P_A\) y \(P_C\) son
signficativos. Es decir, según los resultados de este modelo, la
probabilidad de detención y la probabilidad de condena son
significativas para explicar la tasa de criminalidad. Finalmente, las
estimaciones de los parámetros en valor absoluto cumplen con el
ordenamiento de los efectos disuasorios del modelo de crimen donde
\(P_A>P_C>P_P\).
>>>>>>> 757fa87904969bc08d68db32967ed273e724e447

\hypertarget{posibles-problemas-de-heterogeneidad-no-observable-y-el-estimador-de-efectos-fijos-fe}{%
\subsection{2. Posibles problemas de heterogeneidad no observable y el
estimador de efectos fijos
(FE)}\label{posibles-problemas-de-heterogeneidad-no-observable-y-el-estimador-de-efectos-fijos-fe}}

Existe el riesgo de que los atributos no-observables de los condados
<<<<<<< HEAD
estén correlacionados con \((X'_{it},P'_{it})\) y con \(R_{it}\),
generando estimadores sesgados. Por ejemplo, imaginemos que las tasa de
reporte del crimen varía a nivel de condado y que es no-observable (lo
cúal es bastante plausible). En este caso, aquellos condados con mayores
tasas de subreporte tendrán (en promedio) una menor tasa de criminalidad
y mayores tasas de arresto, a pesar de haber controlado previamente por
las demás variables. Una situación así sesgaría la estimación de los
efectos de interés, haciéndolos parecer de mayor magnitud.

El estimador de efectos fijos (FE o within) elimina toda heterogeneidad
no-observable a nivel de condado y fija en el tiempo. Esto se logra al
añadir variables binarias ``fijas'' a nivel condado que sustraen su
promedio histórico a todas las demás variables. De esta forma las
variables del modelo estan exentas de diferencias de promedio a este
nivel.

Volviendo al ejemplo anterior, un estimador FE hubiese corregido las
diferencias en criminalidad y tasas de arresto promedio entre condados
con alto y bajo subreporte, haciéndolos más comparables.

A continuación, aplicamos el test de especificación de Hausman para
verificar la existencia de varaibilidad no-observable a nivel de
condado. La hipótesis nula del test es que la variabilidad no observable
a nivel de condado es ortogonal a las variables explicativas del modelo.
El estadístico se construye a partir de las diferencias entre los
=======
estén correlacionados con la variable dependiente y las variables
explicativas, generando estimadores sesgados. Por ejemplo, imaginemos
que las tasa de reporte del crimen sea distinta en distintos condados y
sea no-observable (lo cual es bastante plausible). En este caso,
aquellos condados con mayores tasas de subreporte tendrán (en promedio)
una menor tasa de criminalidad y mayores tasas de arresto, a pesar de
haber controlado previamente por las demás variables. Es decir, el
subreporte sesgaría la estimación del efecto de interés, haciéndolo
parecer de mayor magnitud.

El estimador de efectos fijos (FE o \emph{within}) elimina toda
heterogeneidad no-observable a nivel de condado que permanezca fija en
el tiempo. Esto se logra al añadir variables binarias ``fijas'' por
condado que sustraen su promedio histórico a todas las demás variables.
De esta forma las variables del modelo estan exentas de diferencias de
promedio a nivel de condado.

Volviendo al ejemplo anterior, un estimador FE hubiese corregido las
diferencias en criminalidad y tasas de arresto promedio entre condados,
haciéndolos más comparables.

A continuación aplicamos el test de especificación de Hausman verificar
la existencia de varaibilidad no observable a nivel de condado. La
hipotesis nula de este test es que la variabilidad no observable a nivel
de condado es ortogonal a las variables explicativas del modelo. El
estadístico se construye a partir de las diferencias entre los
>>>>>>> 757fa87904969bc08d68db32967ed273e724e447
estimadores del modelo BE y del modelo de FE según la siguiente fórmula:

\begin{gather*}
H = ( \hat \beta_{BE} - \hat \beta_{EF})' (\Omega_{EF}-\Omega_{BE})^{-1}  ( \hat \beta_{BE} - \hat \beta_{EF}) \sim \chi^2
\end{gather*}

\begin{table}[!htbp] \centering 
  \caption{Test} 
  \label{} 
\begin{tabular}{@{\extracolsep{2pt}}lD{.}{.}{-3} D{.}{.}{-3} D{.}{.}{-3}} 
\\[-1.8ex]\hline 
\hline \\[-1.8ex] 
 & \multicolumn{3}{c}{\textit{Hausman Test:}} \\ 
\cline{2-4}\\ 
\\[-1.8ex] & \textit{df} & \textbf{Chi-Sq Statistic} & \textit{p}-values \\ 
\hline \\[-1.8ex] 
Between vs & & & \\
  Fixed Effects Models & 16 & 53.591 & 0.000 \\ 
   & & & \\
\hline 
\hline  \\
\end{tabular} 
\end{table}

<<<<<<< HEAD
El test de Hausman nos permite rechazar la hipótesis nula, lo que nos
invita a usar el estimador FE para lidiar con la heterogeneidad no
observable a nivel de condado (ver Tabla 1).
=======
El test de Hausman rechaza la hipótesis nula de que \(\alpha_i\) es
ortogonal \((X'_{it},P'_{it})\), es decir, en base a este test se puede
concluir que la heterogeneidad no observable a nivel de condado es
estadísticamente importante y que conviene usar la estimacion con
efectos fijos.

El efecto de controlar por condado hace que las estimaciones (en valor
absoluto) de las variables de justicia disminuyan. Por ejemplo, para la
variable \(P_A\) la elasticidad estimada disminuyó aproximadamente
\(41\%\), mientras que para la variable \(P_C\) disminuyó \(43\%\)
aproximadamente. A diferencia de las estimaciones between, el
coeficiente \(P_P\) estimado por efectos fijos tiene no sólo el signo
correcto (negativo) sino es significativo estadísticamente. Al igual que
en el caso anterior, la estimación del parámetro \(S\) es no
significativa y de pequeño impacto. Finalmente, se pudo observar que los
efectos disuasorios estimador (en valor absoluto) se pueden ordenar
según el modelo económico del crimen (\(P_A>P_C>P_P\)).
>>>>>>> 757fa87904969bc08d68db32967ed273e724e447

El efecto de controlar por condado hace que las estimaciones (en valor
absoluto) de las variables de justicia disminuyan. Por ejemplo, la
elasticidad estimada de la variables \(P_A\) y \(P_C\) disminuyeron en
un \(41\%\) y \(43%
\), respectivamente. A diferencia de las estimaciones between, el
coeficiente \(P_P\) estimado por efectos fijos tiene el signo correcto
(negativo) y es significativo estadísticamente. Al igual que en el caso
anterior, la estimación del parámetro \(S\) es no significativa y de
baja magnitud. Finalmente, se pudo observar que los efectos disuasorios
estimados se pueden ordenar según el modelo económico del crimen
(\(|P_A|>|P_C|>|P_P|\)).

<<<<<<< HEAD
\hypertarget{probando-el-estimador-de-efectos-aleatorios-re}{%
\section{3. Probando el estimador de efectos aleatorios
(RE)}\label{probando-el-estimador-de-efectos-aleatorios-re}}

Realizamos nuevamente el test de Hausman, pero esta vez comparando los
estimadores del modelo de efectos aleatorios (RE) con los del modelo FE.
Al igual que en el caso anterior, el estadístico nos permite rechazar la
hipótesis nula de que \(\alpha_i\) es ortogonal \((X'_{it},P'_{it})\).
De nuevo, se puede concluir que la heterogeneidad no-observable a nivel
de condado es estadísticamente significativa.

Esto se da debido a que las estimaciones por efectos aleatorios no
controlan efectos individuales (a nivel de condado), sino que los omite
generando los sesgos discutidos anteriormente.

\begin{table}[!htbp] \centering 
  \caption{Test} 
  \label{} 
\begin{tabular}{@{\extracolsep{2pt}}lD{.}{.}{-3} D{.}{.}{-3} D{.}{.}{-3}} 
\\[-1.8ex]\hline 
\hline \\[-1.8ex] 
 & \multicolumn{3}{c}{\textit{Hausman Test:}} \\ 
\cline{2-4}\\ 
\\[-1.8ex] & \textit{df} & \textbf{Chi-Sq Statistic} & \textit{p}-values \\ 
\hline \\[-1.8ex] 
Within vs & & & \\
  Random Effects Models & 16 & 66.189 & 0.000 \\ 
   & & & \\
\hline 
\hline  \\
\end{tabular} 
\end{table}
=======
Al igual que en el caso anterior, el test de Hausman rechaza la
hipótesis nula de que \(\alpha_i\) es ortogonal \((X'_{it},P'_{it})\).
En base a este resultado se puede concluir que la heterogeneidad no
observable a nivel de condado es estadísticamente significativa. Esto se
da debido a que las estimaciones por efectos aleatorios no controlan
efectos individuales (a nivel de condado), sino que los omite generando
sesgos si \((X'_{it},P'_{it})\) no son ortogonales a \(\alpha_i\) y a
\(\epsilon_{it}\).
>>>>>>> 757fa87904969bc08d68db32967ed273e724e447

\hypertarget{comparando-resultados}{%
\section{4. Comparando resultados}\label{comparando-resultados}}

<<<<<<< HEAD
=======
\emph{Insertar tabla Anzony}

>>>>>>> 757fa87904969bc08d68db32967ed273e724e447
\begin{table}[!htbp] \centering 
  \caption{Resultados} 
  \label{} 
\small 
\begin{tabular}{@{\extracolsep{5pt}}lD{.}{.}{-3} D{.}{.}{-3} D{.}{.}{-3} } 
\\[-1.8ex]\hline 
\hline \\[-1.8ex] 
 & \multicolumn{3}{c}{\textit{Dependent variable:}} \\ 
\cline{2-4} 
\\[-1.8ex] & \multicolumn{3}{c}{Crime Rate} \\ 
<<<<<<< HEAD
 & \multicolumn{1}{c}{Between (BE)} & \multicolumn{1}{c}{Within (FE)} & \multicolumn{1}{c}{Random Effects (RE)} \\ 
=======
 & \multicolumn{1}{c}{Between (BE)} & \multicolumn{1}{c}{Within (EF)} & \multicolumn{1}{c}{Random Effects (RE)} \\ 
>>>>>>> 757fa87904969bc08d68db32967ed273e724e447
\\[-1.8ex] & \multicolumn{1}{c}{(1)} & \multicolumn{1}{c}{(2)} & \multicolumn{1}{c}{(3)}\\ 
\hline \\[-1.8ex] 
 $P_{A}$ & -0.648^{***} & -0.385^{***} & -0.415^{***} \\ 
  & (0.088) & (0.033) & (0.030) \\ 
  $P_{C}$ & -0.528^{***} & -0.301^{***} & -0.325^{***} \\ 
  & (0.067) & (0.021) & (0.020) \\ 
  $P_{P}$ & 0.297 & -0.191^{***} & -0.196^{***} \\ 
  & (0.231) & (0.033) & (0.033) \\ 
  S & -0.236 & 0.026 & 0.019 \\ 
  & (0.174) & (0.025) & (0.026) \\ 
 \hline \\[-1.8ex] 
Observations & \multicolumn{1}{c}{90} & \multicolumn{1}{c}{630} & \multicolumn{1}{c}{630} \\ 
R$^{2}$ & \multicolumn{1}{c}{0.880} & \multicolumn{1}{c}{0.425} & \multicolumn{1}{c}{0.566} \\ 
Adjusted R$^{2}$ & \multicolumn{1}{c}{0.846} & \multicolumn{1}{c}{0.310} & \multicolumn{1}{c}{0.551} \\ 
F Statistic & \multicolumn{1}{c}{25.412$^{***}$ (df = 20; 69)} & \multicolumn{1}{c}{24.220$^{***}$ (df = 16; 524)} & \multicolumn{1}{c}{793.199$^{***}$} \\ 
\hline 
\hline \\[-1.8ex] 
\textit{Note:}  & \multicolumn{3}{l}{$^{*}$p$<$0.1; $^{**}$p$<$0.05; $^{***}$p$<$0.01} \\ 
\end{tabular} 
\end{table}

<<<<<<< HEAD
En la tabla 1 se observa que las estimaciones por FE y por RE (segunda y
tercera columna) de \(P_A\), \(P_C\) y \(P_P\) muestran los signos
(negativos) y el ordenamiento propuesto por el modelo económico de
crimen (\(|P_A|>|P_C|>|P_P|\)). Pos parámetros \(P_A\), \(P_C\) y
\(P_P\) son significativos en ambas estimaciones. Por otro lado, el
coeficiente de \(S\) muestra signo positivo y no es estadísticamente
significativo.

Así también, observa que las elasticidades de \(P_A\), \(P_C\) y \(P_P\)
son mayores en los modelos BE y RE que en el modelo FE. Por ejemplo, se
observa que la elasticidad de \(P_A\) en el modelo FE se reduce en un
\(7.23\%\) respecto a la obtenida con RE, mientras que para \(P_C\) lo
hace en \(7.38\%\) y para \(P_C\) en \(2.55\%\). Las diferencias de
magnitud entre la estimación BE y FE fueron planteadas en el inciso 2.

\emph{INSERTAR ULTIMA TABLA ACA}
=======
En la tabla 1 se observa que al igual que las estimaciones por efectos
fijos, las estimaciones por efectos aleatorios de \(P_A\), \(P_C\) y
\(P_P\) cumplen con los signos (negativos) y el ordenamiento de los
efectos disuasorios (excluyendo \(S\)) que afirma el modelo económico de
crimen (\(P_A>P_C>P_P\)). En ambas estimaciones se pudo observar que los
parámetros \(P_A\), \(P_C\) y \(P_P\) son significativos.

Así también, observa que las elasticidades de \(P_A\), \(P_C\) y \(P_P\)
son mayores en los modelos between y efectos aleatorios que en el modelo
within. Por ejemplo se observa que la elasticidad de \(P_A\) en el
modelo within se reduce en un \(7.23\%\) respecto a la obtenida con
efectos aleatorios, mientras que para \(P_C\) lo hace en \(7.38\%\) y
para \(P_C\) en \(2.55\%\). Las diferencias de magnitud entre la
estimación between y within fueron planteadas en el inciso 2.

Al igual que en el modelo de efectos fijos, el signo de \(S\) en los
resultados de efectos aleatorios no coincide con lo estipulado en el
modelo económico de crimen: es pequeño y no es significativo a nivel
estadístico.
>>>>>>> 757fa87904969bc08d68db32967ed273e724e447

Tal y como indican Cornwell y Trumbull, omision de efectos por condado
generan grandes diferencias en las estimaciones (en este caso entre
between, within y efectos aleatorios). Dados los resultados del test de
Hausman se concluyó que la heterogeneidad es estadísticamente importante
en esta muestra, podemos concluir las estimaciones between y efectos
aleatorios sobreestiman el efecto estimado de las variables de law
enforcement, y que el modelo within es mas conveniente para anular este
sesgo.

\hypertarget{comentarios-sobre-la-presencia-de-efectos-aleatorios-y-correlacion-serial-de-primer-orden}{%
\section{5. Comentarios sobre la presencia de efectos aleatorios y
correlacion serial de primer
orden}\label{comentarios-sobre-la-presencia-de-efectos-aleatorios-y-correlacion-serial-de-primer-orden}}

<<<<<<< HEAD
Para evaluar la presencia de efectos aleatorios se estimó un modelo a
traves de mco y se utilizó el test de Breusch-Pagan. La hipótesis nula
es la no existencia de efectos aleatorios (\(H_0: \sigma^2_\mu=0\)),
testeando la presencia de heterocedasticidad en los erroes del modelo de
pooled. Los resultados rechazan la hipotesis nula; por lo tanto, existen
efectos aleatorios, los errores son heterocedasticos.

=======
>>>>>>> 757fa87904969bc08d68db32967ed273e724e447
\begin{table}[!htbp] \centering 
  \caption{Test} 
  \label{} 
\begin{tabular}{@{\extracolsep{2pt}}lD{.}{.}{-3} D{.}{.}{-3} D{.}{.}{-3}} 
\\[-1.8ex]\hline 
\hline \\[-1.8ex] 
 & \multicolumn{3}{c}{\textit{Breusch-Pagan Test:}} \\ 
\cline{2-4}\\ 
\\[-1.8ex] & \textit{df} & \textbf{BP} & \textit{p}-values \\ 
\hline \\[-1.8ex] 
Between vs & & & \\
  Linear Model & 20 & 188.15 & 0.000 \\ 
   & & & \\
\hline 
\hline  \\
\end{tabular} 
\end{table}

<<<<<<< HEAD
Teniendo en cuenta la siguiente definición de los errores:

\begin{gather*}
u_{it} = \mu_i+ \epsilon_{it} \\
u_{it-1} = \mu_i+ \epsilon_{it-1}
\end{gather*}

Se hace el test de autocorrelación serial sobre el modelo de efectos
aleatorios y se rechaza la hipótesis nula. Por lo tanto, este modelo
presenta autocorrelación. Esto puede estar relacionado con la omisión de
los efectos fijos que genera que los errores \(u_{it}\) y \(u_{it-1}\)
estén correlacionados generando un problema de sesgo.

También se hace el test de autocorrelación serial sobre el modelo de
efectos fijos y se rechaza la hipótesis nula. Por lo tanto, el modelo
presenta autocorrelación serial. Esto sugiere que los errores
\(\epsilon_{it}\) y \(\epsilon_{it-1}\) esten correlacionados.
=======
Para evaluar la presencia de efectos aleatorios se estimó un modelo a
traves de mco y se utilizo el test de Breusch-Pagan. La hipotesis nula
es la no existencia de efectos aleatorios (\(H_0: \sigma^2_\mu=0\)),
testeando la presencia de heterocedasticidad en los erroes del modelo de
pooled. Los resultados rechazan la hipotesis nula; por lo tanto, existen
efectos aleatorios, los errores son heterocedasticos.
>>>>>>> 757fa87904969bc08d68db32967ed273e724e447

\begin{table}[!htbp] \centering 
  \caption{Test} 
  \label{} 
\begin{tabular}{@{\extracolsep{5pt}}lD{.}{.}{-3} D{.}{.}{-3} } 
\\[-1.8ex]\hline 
\hline \\[-1.8ex] 
 & \multicolumn{2}{c}{\textit{Durbin-Watson Test:}} \\ 
\cline{2-3}\\ 
\\[-1.8ex] &  \textbf{DW} & \textit{p}-values \\ 
\hline \\[-1.8ex] 
 \textit{Random Effects Model} & 1.357 & 0.000 \\ 
   & & \\
  \textit{Fixed Effects Model} & 1.687 & 0.000 \\ 
   & & \\
\hline 
\hline \\
\end{tabular} 
\end{table}
<<<<<<< HEAD
=======

Teniendo en cuenta la siguiente definición de los errores:

\begin{gather*}
u_{it} = \mu_i+ \epsilon_{it} \\
u_{it-1} = \mu_i+ \epsilon_{it-1}
\end{gather*}

Se hace el test de autocorrelación serial sobre el modelo de efectos
aleatorios y se rechaza la hipótesis nula. Por lo tanto, este modelo
presenta autocorrelación. Esto puede estar relacionado con la omisión de
los efectos fijos que genera que los errores \(u_{it}\) y \(u_{it-1}\)
estén correlacionados generando un problema de sesgo.

Se realiza el test de autocorrelación serial sobre el modelo de efectos
fijos y se rechaza la hipótesis nula. Por lo tanto, el modelo presenta
autocorrelación serial. Esto sugiere que los errores \(\epsilon_{it}\) y
\(\epsilon_{it-1}\) esten correlacionados.
>>>>>>> 757fa87904969bc08d68db32967ed273e724e447

\end{document}
