% Options for packages loaded elsewhere
\PassOptionsToPackage{unicode}{hyperref}
\PassOptionsToPackage{hyphens}{url}
%
\documentclass[
]{article}
\usepackage{lmodern}
\usepackage{amssymb,amsmath}
\usepackage{ifxetex,ifluatex}
\ifnum 0\ifxetex 1\fi\ifluatex 1\fi=0 % if pdftex
  \usepackage[T1]{fontenc}
  \usepackage[utf8]{inputenc}
  \usepackage{textcomp} % provide euro and other symbols
\else % if luatex or xetex
  \usepackage{unicode-math}
  \defaultfontfeatures{Scale=MatchLowercase}
  \defaultfontfeatures[\rmfamily]{Ligatures=TeX,Scale=1}
\fi
% Use upquote if available, for straight quotes in verbatim environments
\IfFileExists{upquote.sty}{\usepackage{upquote}}{}
\IfFileExists{microtype.sty}{% use microtype if available
  \usepackage[]{microtype}
  \UseMicrotypeSet[protrusion]{basicmath} % disable protrusion for tt fonts
}{}
\makeatletter
\@ifundefined{KOMAClassName}{% if non-KOMA class
  \IfFileExists{parskip.sty}{%
    \usepackage{parskip}
  }{% else
    \setlength{\parindent}{0pt}
    \setlength{\parskip}{6pt plus 2pt minus 1pt}}
}{% if KOMA class
  \KOMAoptions{parskip=half}}
\makeatother
\usepackage{xcolor}
\IfFileExists{xurl.sty}{\usepackage{xurl}}{} % add URL line breaks if available
\IfFileExists{bookmark.sty}{\usepackage{bookmark}}{\usepackage{hyperref}}
\hypersetup{
  pdftitle={TP1 Econometría Avanzada},
  pdfauthor={Casiano, Denys; Daboin, Carlos y Quispe, Anzony},
  hidelinks,
  pdfcreator={LaTeX via pandoc}}
\urlstyle{same} % disable monospaced font for URLs
\usepackage[left=2.5cm,right=2.5cm,top=1.5cm,bottom=2cm]{geometry}
\usepackage{graphicx,grffile}
\makeatletter
\def\maxwidth{\ifdim\Gin@nat@width>\linewidth\linewidth\else\Gin@nat@width\fi}
\def\maxheight{\ifdim\Gin@nat@height>\textheight\textheight\else\Gin@nat@height\fi}
\makeatother
% Scale images if necessary, so that they will not overflow the page
% margins by default, and it is still possible to overwrite the defaults
% using explicit options in \includegraphics[width, height, ...]{}
\setkeys{Gin}{width=\maxwidth,height=\maxheight,keepaspectratio}
% Set default figure placement to htbp
\makeatletter
\def\fps@figure{htbp}
\makeatother
\setlength{\emergencystretch}{3em} % prevent overfull lines
\providecommand{\tightlist}{%
  \setlength{\itemsep}{0pt}\setlength{\parskip}{0pt}}
\setcounter{secnumdepth}{-\maxdimen} % remove section numbering
\usepackage{booktabs} \usepackage{longtable} \usepackage{array} \usepackage{multirow} \usepackage{wrapfig} \usepackage{float} \floatplacement{figure}{H}

\title{TP1 Econometría Avanzada}
\author{Casiano, Denys; Daboin, Carlos y Quispe, Anzony}
\date{2022-04-09}

\begin{document}
\maketitle

\hypertarget{discusion-del-estimador-between}{%
\subsection{1. Discusion del estimador
between}\label{discusion-del-estimador-between}}

\emph{Discuta el modelo estimado (comente acerca de la validez del
estimador between y potenciales sesgos) y comente los resultados
obtenidos, en particular para las variables de justicia criminal.}

\hypertarget{posibles-problemas-de-heterogeneidad-no-observable}{%
\subsection{2. Posibles problemas de heterogeneidad no
observable}\label{posibles-problemas-de-heterogeneidad-no-observable}}

\emph{Explique por qué es muy posible que la presencia de heterogeneidad
no observable a nivel de condado haga que las estimaciones anteriores
sean sesgadas. }

Dos condados pueden ser diferentes en \emph{confounders} no-observables
(confounders son variables correlacionadas con la variable dependiente y
las variables explicativas.)

Por ejemplo, es plausible que haya distintas tasas de reporte del crimen
entre distintos condados, lo cual es inchequeable ya que las denuncias
son la unica manera de medir el crimen. De ser esto cierto, aquellos
condados similares a los demas en todas las caracteristicas observables
pero con altas tasas de subreporte tendran (en promedio) una menor tasa
de criminalidad y mayores tasas de arresto, lo que sesgaria la
estimacion del efecto de las tasas de arresto sobre la criminalidad,
haciendolo parecer de mayor magnitud.

\emph{Realice una estimación con efectos fijos por condado.}

\emph{Discuta por qué esta alternativa resolvería el problema de sesgo.
}\\
Un estimador de efectos fijos estaria exento de toda la variabilidad
atribuible a los condados, por lo que se esta controlando por todos los
no-observables que varian en este nivel.

Acerca del ejemplo anterior, un estimador de efectos fijos (tambien
llamado \emph{within}) hubiese corregido las bajas tasas de criminalidad
y las altas tasas de arresto de nuestros condados con alto subreporte,
haciendolos comparables al resto.

\emph{Testee formalmente la hipótesis nula de ausencia de efectos fijos.
}

CD: Deberiamos comparar los estimadores del modelo between con los
estimadores del modelo de efecto fijo usando el test de Hausmann.

El test de especificacion de Hausman sirve para detectar si hay algun
problema de heterogeneidad no observada, o variables omitidas, en
nuestra especificacion.

La hipotesis nula del test de Hausman es que matriz de variables
omitidas es ortogonal a la matriz de variables incluidas en el modelo.
\[
H0: D\mu  \perp X
\] \[
H1: D\mu  \perp X
\] La hipotesis tiene mayores chances de ser rechazada a medida que la
diferencia entre \(\hat\beta_{FE}\) y \(\hat\beta_{BE}\) es mayor.

Se rechaza la hipotesis nula, confirmando que la espeficacion del modelo
between padece de sesgo por efectos fijos omitidos.

\emph{A la luz del trabajo de CyT, discuta las principales diferencias
encontradas con las estimaciones anteriores.}

\hypertarget{probando-el-estimador-de-efectos-aleatorios}{%
\section{3. Probando el estimador de efectos
aleatorios}\label{probando-el-estimador-de-efectos-aleatorios}}

\emph{Estime el modelo usando un estimador de efectos aleatorios. }

\emph{Implemente un test de Hausman para comparar los estimadores de
efectos fijos y de efectos aleatorios y comente los resultados
obtenidos}

Seguramente encontraremos que el modelo esta mal especificado sin los
efectos fijos, por la misma razon que en la pregunta 2. En este el
estimador de RE, si bien es eficiente en datos de panel, esta sesgado
por omitir controles relevantes.

\hypertarget{comparando-resultados}{%
\section{4. Comparando resultados}\label{comparando-resultados}}

\hypertarget{comentarios-sobre-la-presencia-de-efectos-aleatorios-y-correlacion-serial-de-primer-orden}{%
\section{5. Comentarios sobre la presencia de efectos aleatorios y
correlacion serial de primer
orden}\label{comentarios-sobre-la-presencia-de-efectos-aleatorios-y-correlacion-serial-de-primer-orden}}

\emph{Evalúe con distintos tests la presencia de posibles efectos
aleatorios y correlación serial de primer orden.} CD: No entiendo esta
pregunta. Estan pidiendo que hagamos un test de autocorrelacion sobre
los residuos de un OLS?

\emph{En particular, comente intuitivamente qué es lo que sugiere el
resultado correspondiente al test de autocorrelación.}

CD: Esto es lo que se: Si hay autocorrelacion en datos de panel,
entonces OLS es ineficiente, y conviene usar RE.

Sin embargo, no conviene usar RE cuendo el se rechaza la hipotesis nula
del test de Hausman: es decir, cuando hay una matriz de efectos fijos
que es ortogonal a las variables explicativas. En ese caso conviene una
especificacion de efectos fijos con errores estandar robustos (algo que
aun no hemos visto en la clase.) La idea de aplicar errores estandar
robustos es ajustar hacia arriba los errores estandar de \(\beta\),
porque estos estan subestimados cuando hay autocorrelacion de los
errores. (Tomar con pinza y confirmar todo esto de los robust std.
errors).

\end{document}
